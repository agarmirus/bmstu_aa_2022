\chapter{Технологическая часть}

\section{Требования к программе}

К программе предъявляется ряд требований:

\begin{itemize}
	\item программа должна предоставить пользователю возможность выбора алгоритма сортировки;
	\item после выбора алгоритма программа принимает на вход длину и целочисленные элементы массива;
	\item на выходе - массивы до и после сортировки и среднее время выполнения алгоритма.
\end{itemize}

\section{Средства реализации}

Для написания программы был выбран язык C. Данный выбор обуславливается высокой скоростью выполнения кода и наличием богатой стандартной библиотеки.

\section{Реализация алгоритмов}

На листингах 3.1 - 3.3 представлены реализации алгоритмов сортировки.

\begin{lstlisting}[label=pancake,caption=Функция блинной сортировки, language=Caml]
void pancake_sort(int *arr, const size_t len)
{
	for (size_t right = len - 1; right > 0; --right)
	{
		size_t max_elem_ind = 0;
		for (size_t j = 1; j <= right; ++j)
			if (arr[j] > arr[max_elem_ind])
				max_elem_ind = j;
		if (max_elem_ind > 0)
		{
			for (size_t k = 0; k < max_elem_ind; ++k, --max_elem_ind)
			{
				int tmp = arr[k];
				arr[k] = arr[max_elem_ind];
				arr[max_elem_ind] = tmp;
			}
		}
		for (size_t k = 0, n = right; k < n; ++k, --n)
		{
			int tmp = arr[k];
			arr[k] = arr[n];
			arr[n] = tmp;
		}
	}
}
\end{lstlisting}
\begin{lstlisting}[label=cocktail,caption=Функция сортировки перемешиванием, language=Caml]
void cocktail_sort(int *arr, const size_t len)
{
	size_t left = 0, right = len - 1;
	while (left < right)
	{
		for (size_t i = left; i < right; ++i)
		{
			if (arr[i] > arr[i + 1])
			{
				int tmp = arr[i];
				arr[i] = arr[i + 1];
				arr[i + 1] = tmp;
			}
		}
		--right;
		for (size_t i = right; i > left; --i)
		{
			if (arr[i - 1] > arr[i])
			{
				int tmp = arr[i];
				arr[i] = arr[i - 1];
				arr[i - 1] = tmp;
			}
		}
		++left;
	}
}
\end{lstlisting}
\begin{lstlisting}[label=selection,caption=Функция сортировки выбором, language=Caml]
void selection_sort(int *arr, const size_t len)
{
	for (size_t i = 0; i < len - 1; ++i)
	{
		size_t min_elem_ind = i;
		for (size_t j = i + 1; j < len; ++j)
			if (arr[j] < arr[min_elem_ind])
				min_elem_ind = j;
		if (arr[i] != arr[min_elem_ind])
		{
			int tmp = arr[i];
			arr[i] = arr[min_elem_ind];
			arr[min_elem_ind] = tmp;
		}
	}
}
\end{lstlisting}

\section{Тестовые данные}

В таблице \ref{tests} приведены тестовые данные для функций сортировки. Все тесты пройдены успешно.

\begin{table}[h]
	\begin{center}
		\caption{\label{tests}Тестовые данные для функций сортировки}
		\begin{tabular}{|c|c|c|}
			\hline
			\textbf{Входной массив} & \textbf{Результат} & \textbf{Ожидаемый результат} \\
			\hline
			17, 23, 1, 0, -5, 4 & -5, 0, 1, 4, 17, 23  & -5, 0, 1, 4, 17, 23 \\
			\hline
			-5, 0, 1, 4, 17, 23 & -5, 0, 1, 4, 17, 23  & -5, 0, 1, 4, 17, 23 \\
			\hline
			23, 17, 4, 1, 0, -5 & -5, 0, 1, 4, 17, 23  & -5, 0, 1, 4, 17, 23 \\
			\hline
			3, 3, 3, 3, 3, 3 & 3, 3, 3, 3, 3, 3 & 3, 3, 3, 3, 3, 3 \\
			\hline
			-3, -3, -3, -3, -3, -3 & -3, -3, -3, -3, -3, -3 & -3, -3, -3, -3, -3, -3 \\
			\hline
			1 & 1 & 1 \\
			\hline
		\end{tabular}
	\end{center}
\end{table}

\section*{Вывод}

В данном разделе были разработаны исходные коды следующих алгоритмов сортировки: блинной, перемешиванием и выбором.

\clearpage
