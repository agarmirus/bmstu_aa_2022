\chapter{Исследовательская часть}

\section{Технические характеристики электронно-вычислительной машины}

Ниже приведены технические характеристики электронной вычислительной машины, на которой было произведено исследование работы программы:

\begin{itemize}
	\item ОС Manjaro Linux x86\_64;
	\item ЦП Intel i7-10510U (8) @ 4.900 ГГц;
	\item ОЗУ 8 ГБ.
\end{itemize}

\section{Время выполнения алгоритмов}

Для измерения процессорного времени выполнения программы был использован загаловочный файл time.h стандартной библиотеки языка C \cite{timeh}. Для построения графиков зависимости времени работы реализаций алгоритмов от длины массива был использован пакет PGFPlots \cite{pgfp}.

В таблицах \ref{tnsort} -- \ref{trsort} приведены результаты исследований алгоритмов сортировки.

\begin{table}[H]
	\begin{center}
		\captionsetup{justification=raggedright,singlelinecheck=off}
		\caption{\label{tnsort}Результаты измерений для неотсортированного массива}
		\begin{tabular}{|c|c|c|c|}
			\hline
			Количество элементов & Блинная, сек & Перемешиванием, сек & Выбором, сек \\
			\hline
			100 & 0.036689 & 0.025143 & 0.016479 \\
			\hline
			200 & 0.111341 & 0.071797 & 0.068694 \\
			\hline
			300 & 0.219065 & 0.130772 & 0.118504 \\
			\hline
			500 & 0.542456 & 0.305166 & 0.285259 \\
			\hline
			700 & 1.020449 & 0.621595 & 0.582602 \\
			\hline
			1000 & 1.987858 & 1.174300 & 1.147254 \\
			\hline
		\end{tabular}
	\end{center}
\end{table}
\begin{table}[H]
	\begin{center}
		\captionsetup{justification=raggedright,singlelinecheck=off}
		\caption{\label{tsort}Результаты измерений для отсортированного массива}
		\begin{tabular}{|c|c|c|c|}
			\hline
			Количество элементов & Блинная, сек & Перемешиванием, сек & Выбором, сек \\
			\hline
			100 & 0.026217 & 0.014462 & 0.014656 \\
			\hline
			200 & 0.096718 & 0.050436 & 0.050770 \\
			\hline
			300 & 0.236083 & 0.113531 & 0.109582 \\
			\hline
			500 & 0.536780 & 0.296318 & 0.299467 \\
			\hline
			700 & 1.007199 & 0.588859 & 0.580525 \\
			\hline
			1000 & 1.930346 & 1.140053 & 1.145935 \\
			\hline
		\end{tabular}
	\end{center}
\end{table}
\begin{table}[H]
	\begin{center}
		\captionsetup{justification=raggedright,singlelinecheck=off}
		\caption{\label{trsort}Результаты измерений для обратно отсортированного массива}
		\begin{tabular}{|c|c|c|c|}
			\hline
			Количество элементов & Блинная, сек & Перемешиванием, сек & Выбором, сек \\
			\hline
			100 & 0.045580 & 0.032039 & 0.013493 \\
			\hline
			200 & 0.098482 & 0.058368 & 0.053884 \\
			\hline
			300 & 0.194850 & 0.134700 & 0.124417 \\
			\hline
			500 & 0.546437 & 0.318207 & 0.316290 \\
			\hline
			700 & 1.001914 & 0.602354 & 0.581368 \\
			\hline
			1000 & 1.968974 & 1.196855 & 1.154253 \\
			\hline
		\end{tabular}
	\end{center}
\end{table}

На рисунках \ref{gnsort} -- \ref{grsort} приведены графики зависимости времени работы реализаций алгоритмов сортировки от длины массива.

\begin{figure}[H]
	\begin{center}
		\begin{tikzpicture}
			\begin{axis}[
				title style={align=center},
				title = {Зависимость времени работы реализаций алгоритмов сортировки\\от длины неотсортированного массива},
				xlabel = {Количество элементов массива},
				ylabel = {Время работы, сек},
				legend pos = north west,
				legend style={font=\tiny}
				]
				\legend{ 
					Блинная сортировка, 
					Сортировка перемешиванием, 
					Сортировка выбором
				};
				\addplot[mark=none, color=red] coordinates
				{
					(100, 0.036689)
					(200, 0.111341)
					(300, 0.219065)
					(500, 0.542456)
					(700, 1.020449)
					(1000, 1.987858)
				};
				\addplot[mark=none, color=green] coordinates
				{
					(100, 0.025143)
					(200, 0.071797)
					(300, 0.130772)
					(500, 0.305166)
					(700, 0.621595)
					(1000, 1.174300)
				};
				\addplot[mark=none, color=blue] coordinates
				{
					(100, 0.016479)
					(200, 0.068694)
					(300, 0.118504)
					(500, 0.285259)
					(700, 0.582602)
					(1000, 1.147254)
				};
			\end{axis}
		\end{tikzpicture}
		\caption{\label{gnsort}График для неотсортированного массива}
	\end{center}
\end{figure}
\begin{figure}[H]
	\begin{center}
		\begin{tikzpicture}
			\begin{axis}[
				title style={align=center},
				title = {Зависимость времени работы реализаций алгоритмов сортировки\\от длины неотсортированного массива},
				xlabel = {Количество элементов массива},
				ylabel = {Время работы, сек},
				legend pos = north west,
				legend style={font=\tiny}
				]
				\legend{ 
					Блинная сортировка, 
					Сортировка перемешиванием, 
					Сортировка выбором
				};
				\addplot[mark=none, color=red] coordinates
				{
					(100, 0.026217)
					(200, 0.096718)
					(300, 0.236083)
					(500, 0.536780)
					(700, 1.007199)
					(1000, 1.930346)
				};
				\addplot[mark=none, color=green] coordinates
				{
					(100, 0.014462)
					(200, 0.050436)
					(300, 0.113531)
					(500, 0.296318)
					(700, 0.588859)
					(1000, 1.140053)
				};
				\addplot[mark=none, color=blue] coordinates
				{
					(100, 0.014656)
					(200, 0.050770)
					(300, 0.109582)
					(500, 0.299467)
					(700, 0.580525)
					(1000, 1.145935)
				};
			\end{axis}
		\end{tikzpicture}
		\caption{\label{gsort}График для отсортированного массива}
	\end{center}
\end{figure}
\begin{figure}[H]
	\begin{center}
		\begin{tikzpicture}
			\begin{axis}[
				title style={align=center},
				title = {Зависимость времени работы реализаций алгоритмов сортировки\\от длины неотсортированного массива},
				xlabel = {Количество элементов массива},
				ylabel = {Время работы, сек},
				legend pos = north west,
				legend style={font=\tiny}
				]
				\legend{ 
					Блинная сортировка, 
					Сортировка перемешиванием, 
					Сортировка выбором
				};
				\addplot[mark=none, color=red] coordinates
				{
					(100, 0.045580)
					(200, 0.098482)
					(300, 0.194850)
					(500, 0.546437)
					(700, 1.001914)
					(1000, 1.968974)
				};
				\addplot[mark=none, color=green] coordinates
				{
					(100, 0.032039)
					(200, 0.058368)
					(300, 0.134700)
					(500, 0.318207)
					(700, 0.602354)
					(1000, 1.196855)
				};
				\addplot[mark=none, color=blue] coordinates
				{
					(100, 0.013493)
					(200, 0.053884)
					(300, 0.124417)
					(500, 0.316290)
					(700, 0.581368)
					(1000, 1.154253)
				};
			\end{axis}
		\end{tikzpicture}
		\caption{\label{grsort}График для обратно отсортированного массива}
	\end{center}
\end{figure}

\section{Вывод}

В данном разделе было измерено время работы реализаций следующих алгоритмов сортировки: блинной, перемешиванием и выбором.

По результатам измерений видно, что сортировка выбором работает быстрее других рассматриваемых алгоритмов. Кроме того, графики зависимости времени работы реализаций алгоритмов сортировки от длины массива показывают, что с увеличением числа элементов время работы всех трех реализаций возрастает по параболе. Следовательно, теоретические и практические результаты измерений совпадают.

\clearpage
