\chapter{Аналитическая часть}

\section{Блинная сортировка}

Единственной допустимой операцией в алгоритме является переворот элементов последовательности до определенного индекса. Процесс можно визуально представить как стопку блинов, которую изменяют путём взятия нескольких изделий сверху и их переворачивания \cite{pancake_sort}.

\section{Сортировка перемешиванием}

Сортировка перемешиванием является модифицированной версией <<пузырька>>. В отличие от сортировки пузырьком, во время выполнения алгоритма программа проходит по массиву слева-направо и справо-налево, по необходимости меняя два соседних элемента местами \cite{cocktail_sort}.

\section{Сортировка выбором}

Основные шаги алгоритма:

\begin{itemize}
	\item найти индекс минимального элемента в массиве;
	\item произвести обмен между полученным и первым неотсортированным элементами;
	\item повторить вышеописанные действия, не затрагивая отсортированные элементы массива.
\end{itemize}

Заметим, что данный алгоритм требует наличия всех исходных элементов до начала сортировки, а элементы вывода порождает последовательно, один за другим \cite{selection_sort}.

\section*{Вывод}

В данном разделе было рассмотрено три алгоритма сортировки: блинная, перемешиванием и выбором.

\clearpage
