\chapter*{Введение}
\addcontentsline{toc}{chapter}{Введение}

Сортировкой называют процесс упорядочивания данных в коллекции по определенному признаку. Такое преобразование обеспечивает оптимизацию работы с агрегатными типами данных. В частности, алгоритмы сортировки позволяют облегчить поиск элемента в коллекции.

Существует великое множество алгоритмов сортировки, отличающихся друг от друга не только последовательностью операций, но и количеством потребляемых ресурсов. Поэтому при разработке программного обеспечения необходимо обращать внимание на такие параметры алгоритмов сортировки, как количество потребляемой памяти и время, затраченное на процесс упорядочивания данных.

Целью данной лабораторной работы является получение навыка оценки трудоемкости и временной
эффективности на материале алгоритмов сортировки.

Задачи лабораторной работы:

\begin{itemize}
	\item изучить и реализовать три алгоритма сортировки - блинную, перемешиванием и выбором;
	
	\item выполнить оценку трудоемкости рассматриваемых алгоритмов сортировки;
	
	\item провести замеры процессорного времени работы реализаций
	алгоритмов для данных, соответствующих лучшему, худшему и произвольному
	случаям по трудоемкости;
	
	\item провести сравнительный анализ временной эффективности алгоритмов сортировки.
\end{itemize}
\clearpage
